% !TEX program = pdflatex
\documentclass{article}

\usepackage{hyperref}
\usepackage[round]{natbib}
\usepackage{graphicx}
\usepackage{newtxtext,newtxmath}
\usepackage[UTF8]{ctex}

\author{XingYu Hu}
\title{通过用户活动检测提高Femtocell基站的能效}
\date{\today}

\begin{document}
\maketitle

\begin{abstract}
    毫微微蜂窝基站在许可频谱中运行并在家庭和办公室内提供改进的蜂窝覆盖,引起了无线行业的极大兴趣。
    因此,预计在不久的将来将广泛部署毫微微蜂窝基站。大规模femtocell部署的一个主要问题是由此产生的大量能耗。
    在本文中,我们通过提出一种新颖的节能程序来解决这个问题,该程序允许毫微微小区基站(BS)在不参与活动呼叫时完全关闭其无线电传输和相关处理。结果表明,基于某种语音流量模型,所提出的过程引入了毫微微蜂窝基站功耗的平均减少约37.5%。此外,对于高毫微微小区业务场景,与固定导频发射功率策略相比,还可以实现移动性事件发生的五倍减少。
\end{abstract}

\section{导言}
由于高数据速率的要求和家庭和办公室使用的覆盖范围的改善,femtocell已经引起了无线行业的极大兴趣。住宅和企业femtocell基站(BS)为网络运营商提供了很大的余地,以利用其卓越的室内性能,并为最终用户提供“增值”服务和应用。最终结果是双赢;从用户的角度来看,满意度较高,而网络运营商则受益于CaPeX减少,宏蜂窝流量负担减轻和收入增加。\\

Femtocell是低功耗,低成本,用户部署的蜂窝BS,典型覆盖范围为数十米。它们在许可频谱中运行,并利用用户现有的宽带互联网接入(例如,数字用户线(DSL),有线互联网等)作为回程。由于客户提供的所有内容,运营商的部署,能源供应或现场租赁不会产生额外成本。为了最大限度地降低运营成本,femtocell具有广泛的自动配置和自我优化功能,可实现简单的即插即用部署,并在现有宏蜂窝网络中实现高效集成。\\

深入研究网络部署策略:当前的宏小区方法涉及由网络运营商部署在战略位置的复杂且高度可靠的BS。
相比之下,由于femtocell BS由最终用户安装,因此femtocell部署场景表现出更多随机性。
因此,在以随机即插即用模式为特征的这种情况下难以遵循精确的预推出网络规划的概念。
除了毫微微蜂窝基站的任意部署问题外,还有许多重要的部署后运营考虑因素:
\begin{itemize}
    \item 毫微微蜂窝基站的能源消耗和环境可持续性将成为未来的紧迫问题。据ABI Research [2]称,截至2012年底,预计全球共有超过3600万个毫微微蜂窝基站(销量增长)。假设每个毫微微蜂窝基站需要12W(105.12kWh /年)的功率,所有毫微微蜂窝基站的总能量消耗将达到3.784×109kWh /年。根据电力的生产方式,femtocell的运行每年将产生205万吨二氧化碳。
    因此,为了减少femtocell的能源支出,需要有效的方法来降低能耗而不牺牲其核心优势和功能。
    \item 尽管部署在室内以改善信号接收,但是毫微微蜂窝基站的导频信号可以在其部署的场所外辐射,因此提供不期望的外部覆盖。在拥有大量用户的繁忙地区进行同频道部署,情况变得更加严峻。
    结果,通过的用户被毫微微蜂窝捕获,这可能导致核心网络信令的显着增加。
    虽然毫微微蜂窝覆盖范围的自我优化可以在一定程度上防止这种不希望的影响,但是仍然可能发生不完整的室内覆盖,尤其是如果毫微微蜂窝基站位于次优位置。
\end{itemize}

据我们所知,目前还没有用于控制毫微微蜂窝BS中的空闲或睡眠模式行为的可比解决方案。
为了比较,可以考虑一种简单的基于定时器的解决方案,当在范围内没有检测到移动1时,该解决方案周期性地关闭毫微微小区导频传输一段固定的时间间隔。
然而,该解决方案具有若干缺点,因为当移动设备位于范围内时这是无效的,这是典型家庭中的主要情况。
因此,只有在没有注册用户在覆盖范围内时,毫微微蜂窝基站的能量消耗和场所外的不希望的覆盖范围才会减少。\\

\includegraphics[width=\textwidth]{fig_1.png}

本文通过在毫微微蜂窝基站中启用空闲模式过程来提供针对上述问题的动态节能解决方案,而不管注册用户的位置如何。我们使用术语IDLE来描述毫微微蜂窝基站使用所提出的技术关闭其无线电传输时的操作模式,以及当它们被接通时的ACTIVE。所提出的机制允许在不支持活动呼叫时始终完全关闭毫微微小区传输和相关处理。这显着降低了毫微微蜂窝基站的能量消耗并捕获了不希望的外部用户。\\

本文的其余部分安排如下。第二节简要介绍了femtocell硬件设计和相关的能耗。在第三节中,给出了所提算法的详细描述。第四节评估了算法的性能。最后,结论在第五节中得出。\\

\section{预备知识}

图1示出了典型的毫微微蜂窝基站硬件设计的高级示意图。\\

它包括一个微处理器,负责实现和管理标准化无线电协议栈和相关的基带处理。
一个或多个随机存取存储器组件连接到微处理器,这是各种数据处理功能所需的。
该设计还包含现场可编程门阵列(FPGA)和一些其他集成电路,以实现许多功能,如数据加密,硬件认证,网络时间协议(NTP)等.FPGA中的无线电组件充当微处理器和射频(RF)收发器之间的接口。
存在用于分组发送和接收的单独RF分量,每个分量消耗一定量的功率。还存在RF功率放大器(PA)以将高功率信号传递到发射天线。\\

\includegraphics[width=\textwidth]{tab_1.png}

表I显示了图1中突出显示的毫微微蜂窝基站硬件组件的能耗曲线。完全激活后,硬件电路总共消耗PACT = 10.2 W,电源功率效率为85%。\\

\section{算法说明}
我们建议在正常的fem-tell操作中增加IDLE(空闲)模式,当其注册用户没有进行活动呼叫时,它会禁用其导频传输和相关的无线电处理。
为了启用IDLE模式过程,需要底层宏小区覆盖,因为它依赖于检测从UE到宏小区的传输。
毫微微小区BS包括“低功率嗅探器”能力,其允许检测从移动台到底层宏小区的活动呼叫。\\

当位于毫微微小区的覆盖范围内的移动台向宏小区呼叫时,嗅探器检测到上行链路频带上的接收功率的上升。
如果接收信号强度超过预定阈值,则认为检测到的移动电话足够接近以便可能被毫微微小区覆盖。
噪声基底的上升很容易检测,因为移动设备以高功率发送到宏蜂窝,同时位于非常接近毫微微蜂窝的位置。
图2中示出了这种情况的一个示例。\\
\includegraphics[width=\textwidth]{fig_2.png}
当没有用户参与活动呼叫时,该技术允许毫微微小区BS关闭所有导频传输以及与无线接收相关联的处理。
仅保持将毫微微蜂窝连接与核心网络保持活动所需的硬件组件。回程连接保持活动状态以保持RF同步并避免冗长的启动时间。
一旦从注册的UE检测到活动呼叫,毫微微小区就可以重新激活空中接口及其导频功率传输(ACTIVE模式)。
在激活ACTIVE模式时,移动站将毫微微小区导频报告给它所连接的宏小区,并且移动站从宏小区切换到毫微微小区(如果允许其接入毫微微小区)。\\

基于到宏小区的路径损耗自动配置检测阈值,使得捕获在预期的毫微微小区覆盖范围的边缘处的移动台。
这是必要的,因为移动设备与宏小区建立呼叫所需的发射功率取决于其到宏小区的路径损耗。
对于初始化,可以使用毫微微小区的测量,由于毫微微小区覆盖的小尺寸是一个有用的起点。在操作期间,从移动设备报告宏小区的典型路径损耗。\\
\includegraphics[width=\textwidth]{fig_3.png}
图3示出了包含空闲模式过程的毫微微小区BS的操作流程图。\\

最初,毫微微小区BS驻留在空闲模式,其中导频传输和处理被关闭。在这种状态下,嗅探器忙于对宏小区上行链路频带进行测量。
这些测量允许毫微微小区基于如前所述的上行链路信号强度指示来检测在其覆盖范围内进行呼叫的UE。
如果检测到活动UE,则毫微微蜂窝基站从空闲模式切换到主动模式并激活其处理和导频信号传输。
毫微微小区范围内的活动移动站现在将毫微微小区导频测量报告给宏小区。
如果允许UE接入毫微微小区,则将启动UE的宏小区到毫微微小区切换;否则毫微微蜂窝基站恢复到空闲模式。
在完成切换过程后,毫微微蜂窝基站为移动台服务直到其呼叫完成。
在完成呼叫之后,毫微微蜂窝基站切换回IDLE模式并禁用处理和导频信号传输。\\

毫微微小区BS中提出的IDLE模式切换需要在毫微微小区切换其导频时从毫微微小区注册的移动站切换一个宏小区到毫微微小区,
并且当毫微微小区再次切换其导频时需要一个毫微微小区到宏小区小区重选过程(假设不同的位置区域代码用于毫微微小区和宏小区网络)。
但是,正如我们在下一节中所看到的,由于关闭的导频传输导致的移动性事件减少和相关信令的好处超过了这一现象。\\

\section{数值结果}
在本节中,我们通过检查femtocell功耗和核心网络信令的减少来评估所提出的IDLE模式的性能。\\
\subsection{降低功耗}
令$ P_ {saved} $是通过在毫微微小区BS的空闲模式中关闭硬件组件而节省的功率。
因此,IDLE模式下的功耗等于具有$\overline{\eta}$的毫微微蜂窝基站,毫微微蜂窝基站的$P_{I D L E}=P_{A C T}-P_{s a v e d}$的平均减少。
通过表示平均占空比,功耗可以表征为总功率的百分比:$$
\Omega=\frac{P_{\text {saved}}}{P_{A C T}} \times(1-\overline{\eta}) \times 100
$$
通过切换到IDLE模式,毫微微蜂窝BS关闭PA,RF发送器,RF接收器以及与IDLE模式中的非必要功能相关的各种硬件组件,例如数据加密,硬件认证等。
接通低功率无线电嗅探器($P_{\text {snif} f}=0.3 \mathrm{W}$)以在宏小区上行链路频带上执行接收功率测量。
根据表I中提供的硬件额定值,可以节省以下功效:$$
\begin{aligned} P_{\text {saved}} &=P_{P A}+P_{T X}+P_{R X}+P_{\text {misc}}-P_{\text {sniff}} \\ &=2.0+1.0+0.5+1.0-0.3=4.2 \mathrm{w} \end{aligned}
$$
其中$\Omega$最大是$\left(\frac{4.2}{10.2}\right) \cdot 100=41.2 \%$。
\end{document}
