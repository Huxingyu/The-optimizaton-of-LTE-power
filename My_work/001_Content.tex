% !TEX program = pdflatex
\documentclass{article}

\usepackage{hyperref}
\usepackage[round]{natbib}
\usepackage{newtxtext,newtxmath}
\usepackage[UTF8]{ctex}

\author{XingYu Hu}
\title{对于LTE节能算法的研究の方法综述}
\date{\today}

\begin{document}
\maketitle
前一阵子查找了不少的文献,对于LTE基站的节省算法也有了一定的了解;但是LTE基站的节能算法的选择是十分的多的,
算法说到底还是数学嘛。\\

开门见山,LTE的节能主要靠两个方面,单小区的频率调节和多小区的智能选择,而后者在节能的占比上达到了80\%,
所以我们主要关注智能选择这个课题。为什么要智能选择呢,因为通信网络的潮汐效应,通俗来说就是晚上睡了不关灯。
而智能选择的意思是不需要的或者用户少的时候,给这个小区较少的资源,你不关灯我帮你关。
(人们急需用网的时候关闭资源,即所谓的人工智障)\\


高大上一些来说,本课题是基于目前异构密集通信网络中的能效问题,研究通过基站休眠策略达到提高通信网络
能效的目的。根据移动通信网络负荷在时间上存在不连续性和不均衡性的特点,结合LTE技术特点,研究如何自动、动态地
调节无线资源的使用,从而达到降低能耗的目的。\\

下面给出要学习的若干文献:\\
1,\underline{博弈论}《基于博弈论的LTE基站自优化节能方法研究和试验》\\
2,\underline{自组织}《基于自组织网络的LTE基站节能优化解决方案》\\
3,\underline{\textbf{神经网络}}《密集网络下基于能效优化的基站睡眠控制策略研究》\\

另外还有一些比较有意思的文献,是解决资源分配的,等这个课题结束了可以玩玩:\\
1,\underline{图论} 《家庭基站系统中基于图论的资源分配算法研究》 \\
2,\underline{模拟退火算法,蚁群算法} 《LTE宏基站与家庭基站的弹性接入管理节能机制研究》

今天先搞到这里,晚安济南。
\end{document}
